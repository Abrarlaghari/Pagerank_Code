\documentclass{article}

\usepackage[tight]{subfigure}
\usepackage[dvips]{color}
\usepackage[table]{xcolor}

\usepackage[unicode, bookmarks=false]{hyperref}
\usepackage{amsfonts, amsthm, graphicx, bbm}

%------- Some definitions -----------------------------------------------------
\def\note#1{\textcolor{red}{({\bf note:} \em\small #1)}}
\def\question#1{\textcolor{blue}{\em\small #1?}}


\def\eq#1{Eq.~(\ref{#1})}
\def\eqs#1#2{Eq.~(\ref{#1})--(\ref{#2})}
\def\fig#1{Figure~\ref{#1}}
\def\tabl#1{Table~\ref{#1}}
\def\alg#1{Algorithm~\ref{#1}}
\def\se#1{Section~\ref{#1}}
\def\vect#1{\mbox{$\bf #1$}}
\def\mx#1{\mbox{$\bf #1$}}

\newcommand{\eg}{{\em e.g.\ }}
\newcommand{\ie}{{\em i.e.\ }}
\newcommand{\etal}{ {\em et al.\ }}


\title{TITLE}
\author{Authors}


\begin{document}
\maketitle

\begin{abstract}

ABSTRACT
\end{abstract}

\section{Introduction}\label{sec:1}

\noindent\textbf{Winning percentage.}

\noindent\textbf{Rating percentage index.}

\noindent\textbf{Least squares method.}

\noindent\textbf{Colley method.}

\noindent\textbf{Keener method.}

\noindent\textbf{\'El\H{o} method.}

\noindent\textbf{PageRank method.}

\section{Related work}

\section{Ranking methods}

In this section, we give a short description of the ranking methods we will use. The methods will be compared in Sec. \ref{sec: Results}. A more detailed introduction about ranking methods, we refer to \cite{} and \cite{}. Before going into the details, we need some definitions.

Let $V=(1, \dots, n)$ be the set of $n$ teams to be rated and let $R$ be the number of rounds in a competition of the teams in $V$. After round $r$ ($r=1, \dots, R$), a \textit{rating} $\phi^{r}: V \rightarrow \mathbb{R}$ assigns a score to each team which we call quantitative ``strength". A ranking $\sigma: V\rightarrow V$ is an ordering of the teams, which we can simply obtain from a rating on $V$ by sorting. For rating and ranking the teams, we consider the following two data sets: (i) the game results as win, lose or draw; (ii) the final scores of the games.

Let us define the following matrices $W, D, S \in \mathbb{R}^{n \times n}$ as
\begin{equation}\label{eq: Wmatrix}
W_{ij}=\#\{i \textrm{ won against } j\},
\end{equation}
\begin{equation}\label{eq: Dmatrix}
D_{ij}=\#\{\textrm{draws between } i \textrm{ and } j\},
\end{equation}
\begin{equation}\label{eq: Smatrix}
S_{ij}=\sum_{\textrm{game btwn } i \textrm{ and } j} \frac{\# \textrm{ points } i \textrm{ scored ag/ } j}{\textrm{total points scored}}.
\end{equation}
Note, that $W$ is obtained by the (i) data set, while $S$ is obtained by the (ii) data set. In eq. \ref{eq: Smatrix} we take $1/2$ if the outcome of a game is $0:0$. The elements of the vectors $\vect{w}=W\mathbb{\vect{1}}$, $\vect{l}=W^{t}\mathbb{\vect{1}}$, $\vect{d}=D\mathbb{\vect{1}}$  and $\vect{t}=(W+W^{t}+D)\mathbb{\vect{1}}$ are the number of wins, loses, draws and total number of games played by team $i$ ($i=1\dots, n$), respectively. Since each game is either a win, a lose, or a draw, thus $\vect{t}=\vect{w}+\vect{l}+\vect{d}$. We define $T=diag(\vect{t_i})$ that is the diagonal matrix with entries $T_{ii}=t_i$ ($i=1\dots, n$). Similarly, we define the vectors $\vect{s}=S\mathbb{\vect{1}}$, $\vect{u}=S^{t}\mathbb{\vect{1}}$, and it is easy to see, that $\vect{s}+\vect{u}=\vect{t}$



\subsection{Winning percentage (WP)}
The winning percentage of a team $i$ after round $r$ is simply defined as $\phi^{r}_{WP, i}= (w_i+\kappa d_i)/t_i$, where $\kappa$ is parameter between $0$ and $1$ and can be interpreted as the ``value" of a draw. For example, we take $\kappa=1/3$, it refers that that the value of a draw is one third of the value of a win. The vector of winning percentages after round $r$ can be computed as 
\begin{equation}\label{eq: wp}
\phi^{r}_{WP}=T^{-1}(\vect{w}+\kappa\vect{d}).
\end{equation}
By considering the scores of of the games (i.e. by using the data set (ii)), a similar quantity can be calculated as
\begin{equation}\label{eq: wpS}
\phi^{r}_{WP-S}=T^{-1}\vect{s}.
\end{equation}
We highlight, that the methods \ref{eq: wp} and \ref{eq: wpS} do not take into consideration the strength of the opponent teams, only the outcome of the single games counts.

\subsection{Rating percentage index (RPI)}

The Rating Percentage index takes into account the WPs of its opponents, and its opponents' opponents \cite{}. The average winning percentage of team $i$'s opponents after round $r$ is calculated as
\begin{equation}\label{eq: avg-WP}
\langle N^r(i) \textrm{ winning percentage}  \rangle = \frac{1}{t_i}\sum_{j} (W_{ij}+W_{ji}+D_{ij})\phi^{r}_{WP, j},
\end{equation}
where $N^r(i)$ denotes the set of team previous opponents after round $r$. We obtain that the vector of the average opponents' winning percentages is $T^{-1}(W+W^t+D)\phi^{r}_{WP}$. The winning percentages of the opponents' opponents can be calculated as $T^{-1}(W+W^t+D)^2\phi^{r}_{WP}$ and then, after round $r$, RPI vector is calculated as the following weighted average:
\begin{equation}\label{eq: rpi}
\phi^{r}_{RPI}=\frac{1}{4}\phi^{r}_{WP}+\frac{1}{2}T^{-1}(W+W^t+D)\phi^{r}_{WP}+\frac{1}{4}T^{-1}(W+W^t+D)^2\phi^{r}_{WP},
\end{equation}
and similarly, in case of the more detailed data set (ii)
\begin{equation}\label{eq: rpiS}
\phi^{r}_{RPI}=\frac{1}{4}\phi^{r}_{WP-S}+\frac{1}{2}T^{-1}(W+W^t+D)\phi^{r}_{WP-S}+\frac{1}{4}T^{-1}(W+W^t+D)^2\phi^{r}_{WP-S}
\end{equation}

\subsection{Massey least squares method (M)}

The only statistics is used by Massey's least squares method \cite{} are the number of wins and losses of each time. The rating of the teams after round $r$ is obtained by the solution of the linear system
\begin{equation}\label{eq: Mas}
M\phi^{r}_{M}=\vect{w}-\vect{l}
\end{equation}
where $M=T-W-W^t-D$ contains the total number of games of the teams in the diagonal, and an $(i,j)$ entry of it contains the number of games played between teams $i$ and $j$. The method naturally incorporates with draws, since a draw between two teams increases $M_{ij}$ and $M_{ji}$ by one, while the right hand-side of eq. \ref{eq: Mas} remains unchanged. Since $rank(M)<n$, the linear system \ref{eq: Mas} does not have a unique solution. To overcome this problem, one possible solution is to replace any row in $M$ with a row with all ones and the corresponding entry of $w-l$ with zero.


\subsection{Colley method (C)}

The Colley method is also a modification of the least squares method by using an observation called Laplace’s rule of
succession (see \cite, page 148) which states, that if one observed $k$ successes out of $r$ attempts, then $(k+1)/(r+1)$ is better estimation for the next event to be success than $k/n$. The rating vector of the teams is the solution of the linear system
\begin{equation}\label{eq: Col}
C\phi^{r}_{C}=\vect{b},
\end{equation}
where $C=M+2I$ (here, $I$ is the identity matrix) and $\vect{b}=\mathbb{\vect{1}}+1/2(\vect{w}-\vect{l})$. It can be easily seen that the linear system \ref{eq: Col} always has a unique solution. 

\subsection{Keener method (K)}

Keener's method is a so-called spectral rating method which uses the Perron-Forbenius eigenvector for the rating and (after round $r$) it is given by the solution of the eigenvalue equation
\begin{equation}\label{eq: Ken}
T^{-1}(W+\kappa D)\phi^{r}_{K}=\lambda \phi^{r}_{K}
\end{equation}
in case when the scores are not considered. In eq. \ref{eq: Ken}, $\lambda$ is the dominant eigenvalue of the matrix $T^{-1}(W+\kappa D$ and it exists for a matrix with non-negative entries such that any other eigenvalue is smaller in absolute value. The corresponding eigenvector (called the Perron-Frobenius eigenvector) has non-negative entries and now it gives the rating of the teams.

Originally, the method was defined for the case when we consider the scores of the games. The Keener matrix, also based on the Laplace's rule of succession, is defined as
\begin{equation}\label{eq: K}
K_{ij}=h\bigg(\frac{S_{ij}+1}{S_{ij}+S_{ji}+1}\bigg)
\end{equation}
where $h$ is a skewing function helping to reduce the difference between the upper and lower ends of the rating. We use the original function defined by Kenner, namely,
\begin{equation}\label{eq: h}
h(x)=\frac{1}{2}+\frac{1}{2}\textrm{sgn}(x-\frac{1}{2})\sqrt{|2x-1|}
\end{equation}
The Keener rating vector of the teams is given by the solution of the equation
\begin{equation}\label{eq: Ken}
T^{-1}K\phi^{r}_{K-S}=\lambda \phi^{r}_{K-S}
\end{equation}

\subsection{\'El\H{o} method (E)}

Let $\phi^{r-1}_{E,i}$ and $\phi^{r-1}_{E,j}$ be the \'El\H{o} ratings of teams $i$ and $j$, resp, after round $r-1$. For each game in round $r$, the \'El\H{o} method computes the expected outcome of the game as a logistic function of the difference of the team ratings. This function is $L(x)=1/(1+10^{-x/\xi})$, where $\xi>0$ is a parameter and the expected outcome of the game between $i$ and $j$ in round $r$ is
\begin{equation}\label{eq: Elo}
E^r_{ij}=L(\phi^{r-1}_{E,i}-\phi^{r-1}_{E,j}).
\end{equation}
If we consider the outcomes of the matches just as win, lose or draw, 

\subsection{PageRank method (PR)}

\section{Forecasting}

\subsection{The Bradley-Terry model}

\subsection{Betting odds}

\section{Results} \label{sec: Results}

\subsection{Data set}

\subsection{Comparison of the ranking methods}

\subsection{Hypotheses testing}

\section{Discussion}

\section*{Acknowledgment}

\bibliographystyle{acm}      % mathematics and physical sciences
%\bibliographystyle{spphys}       % APS-like style for physics
\bibliography{}{}   % name your BibTeX data base


\end{document}